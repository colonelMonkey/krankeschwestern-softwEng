\documentclass[12pt]{article}
\usepackage[margin=1in]{geometry}
\usepackage{pdfpages}
\usepackage[sfdefault]{GoSans}
\linespread{1.5}


\begin{document}
\includepdf[]{DeckblattPflichtenheft.pdf}
\title{Pflichtenheft}

\author{Die kranken Schwestern}

\tableofcontents
\pagebreak

\section{Zielbestimmung}
\subsection{Musskriterien}
Das Programm soll dazu dienen, Zelluläre Automaten auf einem 2-D orthogonalen Spielfeld darstellen zu können. Dazu werden als Beispiel die Regeln für Conway`s Game of Life verwendet.
Hierzu sind unbedingt die folgenden Features erforderlich:
\begin{itemize}
    \item Nach dem Programmstart soll dem Benutzer ein 80x80 Felder großes Spielfeld präsentiert werden. Unterhalb des Spielfeldes sind verschiedene Knöpfe und Kontrollen anzuordnen, welche im folgenden näher erläutert werden sollen.
    \item Es ist erforderlich, die Simulation zu starten und zu stoppen, hierzu wird ein Button benötigt. 
    \item Es ist erforderlich, eine einzelne Generation weiter springen zu können, dies benötigt ebenfalls einen Button.
    \item Es ist Sinnvoll, die Simulationsgeschwindigkeit einstellen zu können, dies soll in Form eines Sliders geschehen. Die Simulationsgeschwindigkeit soll zwischen 0,5 und 10 Generationen pro Sekunde frei Wählbar sein.
    \item Es soll die Größe des Spielfeldes anpassbar sein. Dies soll so geschehen, dass über einen Button auf dem Hauptinterface ein Fenster aufgerufen wird, auf dem die Größe des Spielfeldes mit zwei Eingabefeldern als int,  sowie Randverhalten eingestellt werden können.
    \item Das Randverhalten soll zwischen zwei Optionen mit RadioButtons oder ähnlichem wählbar sein, sodass das Spiel entweder auf einer endlichen Fläche mit toten Rändern läuft, oder torusartig an den Enden zusammengebogen wird.
    \item Es ist gewünscht, dass die Spielregeln anpassbar sind. Dies soll über eine Eingabezeile geschehen, welche in einem Regeleditor existiert. Der Regeleditor soll über einen Button auf dem Hauptinterface aufrufbar sein.
    \item Es ist hochgradig nützlich, einen Spielstand speichern und laden zu können. Dies soll einfach gehalten werden: es soll der Zustand des Feldes sowie der Zustand des Regeleditors in eine Datei ausgelagert werden.
    \item Zum Laden von Spielständen: Es sollen Spielstände von der Festplatte geladen werden können. Dies soll auch dazu dienen, bereits bekannte Spielfeldkonstruktionen in das aktuelle Spielfeld einzufügen. Das bedeutet, dass es ein "komposit-laden" geben soll: Wird dies ausgewählt UND ist das Spielfeld des zu ladenden Spielstands kleiner als das Aktuelle, so soll per Mausklick das zu ladende Spielfeldkonstrukt das Spielfeld an den entsprechenden Stellen überschreiben und das Konstrukt so in das Spiel integrieren.
    \item Das Spielfeld soll als zweidimensionales Array ausgeführt sein, in welchem Integer-Werte einen Zellzustand festlegen. Dieses soll im Hauptfenster angezeigt werden, wobei die verschiedenen Zellzustände durch Farben angedeutet werden sollen.
    
    
    
    
\end{itemize}
\subsection{Wunschkriterien}
\begin{itemize}
    \item Es ist wünschenswert, einer Farbe einen Zustand zuordnen zu können. Ein einfaches Color ramp kann ggf. verwendet werden, alternativ ist es  
\end{itemize}


\pagebreak
\section{Produkt-Einsatz}
\subsection{Anwendungsbereich}
Das Programm soll dazu dienen, Zelluläre Automaten mit recht großer Freiheit bauen zu können. 
\subsection{Zielgruppen}
Die Verwendung dieses Programms für Conway's Game of life ist einfach, da die Spielregeln mitgeliefert werden. Dies kann von allen interessierten ausprobiert werden, da die Manipulation des Spielfelds zum ausprobieren einlädt.

Leider ist es nicht möglich, den Regeleditor intuitiv bedienbar zu gestalten, da es für eine effiziente Verarbeitung notwendig ist, den Zustand einer Zelle in der nächsten Generation als Mathematische Funktion der Zustönde der Nachbarzellen darzustellen. Aus diesem Grund gibt es zwar einen Leitfaden, um Mathematische Funktionen mit den Umliegenden Zellen als Ausgangsdaten zu erstellen, es ist jedoch nicht einfach, dies zu tun. Deal with it.
\subsection{Produktumgebung}
\subsubsection{Softwareanforderungen}

\subsubsection{Hardwareanforderungen}
\subsection{Betriebsbedingungen}
\pagebreak
\section{Produktfunktionen}
\subsubsection{Benutzeroberfläche}
\subsubsection{Datenverarbeitung}
\subsubsection{Datenspeicherung}
\subsection{Nichtfunktionale Anforderungen}
\subsubsection{Performance}
\subsubsection{Zuverlässigkeit}
\pagebreak
\section{Testszenarien}
\subsection{UI}
\subsection{Verarbeitung}
\subsection{Speichern}
\subsection{Performance}
\subsection {Benutzbarkeit (Schimpanse benötigt)}
\pagebreak
\section{Entwicklungsumgebung}
\subsection{Verwendete Software}
\subsection{Verwendete Hardware}
\subsection{verwendete Organisation}


\pagebreak




\end{document}
